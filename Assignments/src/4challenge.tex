\documentclass[11pt]{article}

%%  Dimensions and URL
\usepackage[margin=1in]{geometry}
\usepackage{hyperref}

%%  Definitions
\renewcommand{\baselinestretch}{1.1}
\pagestyle{plain}


\begin{document}

\begin{center}
{\bfseries \LARGE Programming Challenge 4}
\end{center}

Use the random module to generate random numbers.
For instance, \texttt{random.randint(1,6)} returns a random integer ranging from $[1,6]$.
To use this module, it is necessary to import \texttt{random}.
\begin{verbatim}
import random
\end{verbatim}

In this challenge, you roll a fair six-sided die.
If the result is an odd number, you roll once more then stop; otherwise, you stop right after the first roll.
Compute the probability of the sum of your total rolls.
Let $X$ and $Y$ correspond to the results of your first roll and second roll (if possible), and $Z$ be the sum of your total rolls.
Basically, you are asked to compute the PMF of $Z$.

\begin{verbatim}
NumberTrials = 1000
sequenceX=[]
sequenceY=[]
sequenceZ=[]
\end{verbatim}

If the result of your first roll is an odd number, then you roll once more then stop; if the result of your first roll is an even number, you can assume the result of your second roll is $0$.
(This assumption does not affect the value of $Z$ since you stop right after the first roll.)
\begin{verbatim}
for TrialIndex in range(0, NumberTrials):
    sequenceX.append(random.randint(1,6))
    if sequenceX[TrialIndex] == 1 or sequenceX[TrialIndex] == 3 or sequenceX[TrialIndex] == 5:
        sequenceY.append(random.randint(1,6))
    else:
        sequenceY.append(0)
    sequenceZ.append(sequenceX[TrialIndex] + sequenceY[TrialIndex])
\end{verbatim}

Look at the empirical distribution of $Z$. 
\begin{verbatim}
percent = []
for OutcomeIndex in range(2, 12):
    percent.append(sequenceZ.count(OutcomeIndex) / float(NumberTrials))
print(percent)
\end{verbatim}
 Calculate by hands the probability $Z=4$ and compare with the corresponding empirical result.
Explore how the empirical distribution changes as \texttt{N} grows: 1000, 10000, 100000, etc.
Note that the range for random variable $Z$ should be $\{2, 3, 4, \ldots, 11\}$.
\end{document}

